% -*- TeX:de -*-
\NeedsTeXFormat{LaTeX2e}
\documentclass[12pt,a4paper]{article}
\usepackage[german]{babel} % german text
\usepackage[DIV12]{typearea} % size of printable area
\usepackage[T1]{fontenc} % font encoding
%\usepackage[latin1]{inputenc} % most likely on Windows
\usepackage[utf8]{inputenc} % probably on Linux
\usepackage{multicol}

% PLOTTING
\usepackage{pgfplots} 
\usepackage{pgfplotstable}

\usepackage{graphicx} % to include images
\usepackage{subfigure} % for creating subfigures
\usepackage{amsmath} % a bunch of symbols
\usepackage{amssymb} % even more symbols
\usepackage{booktabs} % pretty tables

% a floating environment for circuits
\usepackage{float}
\usepackage{caption}

\newfloat{circuit}{tbph}{circuits}
\floatname{circuit}{Schaltplan}

% a floating environment for diagrams
\newfloat{diagram}{tbph}{diagrams}
\floatname{diagram}{Diagramm}

\selectlanguage{german} % use german

\begin{document}

%%%%%%% DECKBLATT %%%%%%%
\thispagestyle{empty}
			\begin{center}
			\Large{Fakultät für Physik}\\
			\end{center}
\begin{verbatim}


\end{verbatim}
							%Eintrag des Wintersemesters
			\begin{center}
			\textbf{\LARGE WS 2013/14}
			\end{center}
\begin{verbatim}


\end{verbatim}
			\begin{center}
			\textbf{\LARGE{Physikalisches Praktikum\\ für das Bachelorstudium}}
			\end{center}
\begin{verbatim}




\end{verbatim}

			\begin{center}
			\textbf{\LARGE{PROTOKOLL}}
			\end{center}
			
\begin{verbatim}





\end{verbatim}

			\begin{flushleft}
			\textbf{\Large{Experiment (Nr., Titel):}}\\
							%Experiment Nr. und Titel statt den Punkten eintragen
			\LARGE{PW1, Messen - Messfehler}	
			\end{flushleft}

\begin{verbatim}

\end{verbatim}	
							%Eintragen des Abgabedatums, oder des Erstelldatums des Protokolls
			\begin{flushleft}
			\textbf{\Large{Datum:}} \Large{17.10.2013}
			\end{flushleft}
			
\begin{verbatim}
\end{verbatim}
							%Namen der Protokollschreiber
		\begin{flushleft}
			\textbf{\Large{Namen:}} \Large{Patrick Braun, Johannes Kurz}
			\end{flushleft}

\begin{verbatim}


\end{verbatim}
							%Kurstag und Gruppennummer, zb. Fr/5
			\begin{flushleft}
			\textbf{\Large{Kurstag/Gruppe:}} \Large{DO/2}
			\end{flushleft}

\begin{verbatim}



\end{verbatim}
							%Name des Betreuers, das Praktikum betreute.
			\begin{flushleft}
			\LARGE{\textbf{Betreuer:}}	\Large{}	
			\end{flushleft}

%%%%%%% DECKBLATT ENDE %%%%%%%
\pagebreak
\begin{multicols}{2}
\section{Mechanische Messungen}
\subsection{Messung der Dicke eines Drahtes mit dem Mikroskop} 

Mikroskop: Karl Zeiss, Axiostar plus 5x, 10x, 40x 

Beobachtungen mit Mikroskop ohne Skala\\
Scharfstellen, Höhe eingestellt, größte Vergrößerung stellen.
71 Striche ohne Skala.\\
71 Striche ohne Skala.\\
70 Striche ohne Skala.\\
71 Striche ohne Skala.\\
72 Striche ohne Skala.\\
\\
Beobachtung des Eich-Strichs beginnend am Rand zur Erleichterung. Fixieren des Plättchens. 
40 am Okular entsprechen 0.1mm\\
Ungenauigkeit Instrument: 0.0025/Strich
xy - bei erster Messung\\

\subsection{Messung der Dicke eines Drahtes mit einem Schublehre}
Schublehre sehr ungenau. \\
Messungen: \\
0.2mm\\
0.25mm\\
0.2mm\\
0.25mm\\
0.2mm\\
Ungenauigkeit Instrument: 0.5mm

\subsection{Messung der Dicke eines Drahtes mit einer Micrometerschraube}
Messung Rand unten: 0.18(5)mm Schätzung weil zwischen 0.18 und 0.19 \\
Messung Rand unten: 0.18(5)mm Schätzung weil zwischen 0.18 und 0.19 \\
Messung Rand oben: 0.18mm\\
Messung bei ca. 1/3: 0.17(9)mm\\
Messung Mitte: 0.17(1)mm\\
Ungenauigkeit Instrument: 0.01mm

\subsection{Messung der Winkel eines Dreiecks mit dem Winkellehre}

Metallenes Dreieck. Firma Storm.\\
Messungenauigkeit von 5 Minuten.\\
$\alpha = $ 23 Grad, 40 Minuten\\
$\beta =$ 71 Grad, 15 Minuten\\
$\gamma = $ 84 Grad, 15 Minuten\\

\section{Pendel}
"Aufgabenstellung in einem Satz" - Wir bestimmen die lokale Erdbeschleunigung über den Zusammenhang zwischen Länge und Periodendauer eines Fadenpendels.\\

Über 6.3 Grad entsteht ein Fehler der über der Messunsicherheit liegt. \\
Messung bei Nulldurchgang.
Messunsicherheit 1/100 Sekunden.

\subsection{10 Einzelschwingungen}
1.59s\\
1.59s\\
1.63s\\
1.57s\\
1.58s\\
1.57s\\
1.56s\\
1.59s\\
1.56s\\
1.57s\\
Kugeldurchmesser: 1.995cm $\pm 0.05mm$\\
Seillänge 62.05cm\\
Bis Schwerpunkt 63cm\\
Mittelwert: 1.581, Stabw: 0.0207, StdError (u): 0.00657, Var : 0.00043
\subsection{10 Mehrfachschwingungen}
15.91s\\
15.91s\\
15.89s\\
15.92s\\
15.78s\\
15.94s\\
15.83s\\
15.93s\\
15.90s\\
15.92s\\
15.93s\\
Mittelwert: 1.5808, Stabw: 0.0031, StdError (u): 0.000986, Var : 9.73 $10^{-6}$
\subsection{Auswertung}
Bei längeren Laufzeiten ist mein Messfehler weniger relevant.

Nur mit 10-Facher Schwingungsdauer g berechnen.
g = 9.8379 $ms^{-2}$ $\pm $ 0.00125 $ms^{-2}$
\section{Strom}
\end{multicols}
\end{document}