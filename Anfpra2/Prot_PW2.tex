% -*- TeX:de -*-
\NeedsTeXFormat{LaTeX2e}
\documentclass[12pt,a4paper]{article}
\usepackage[german]{babel} % german text
\usepackage[DIV12]{typearea} % size of printable area
\usepackage[T1]{fontenc} % font encoding
%\usepackage[latin1]{inputenc} % most likely on Windows
\usepackage[utf8]{inputenc} % probably on Linux
\usepackage{multicol}

% PLOTTING
\usepackage{pgfplots} 
\usepackage{pgfplotstable}
\usepackage{url}
\usepackage{graphicx} % to include images
\usepackage{tikz}
\usepackage{subfigure} % for creating subfigures
\usepackage{amsmath} % a bunch of symbols
\usepackage{amssymb} % even more symbols
\usepackage{booktabs} % pretty tables
\usepackage{makecell} % multi row table heading

% a floating environment for circuits
\usepackage{float}
\usepackage{caption}

%\newfloat{circuit}{tbph}{circuits}
%\floatname{circuit}{Schaltplan}

% a floating environment for diagrams
%\newfloat{diagram}{tbph}{diagrams}
%\floatname{diagram}{Diagramm}

\selectlanguage{german} % use german

\begin{document}

%%%%%%% DECKBLATT %%%%%%%
\thispagestyle{empty}
			\begin{center}
			\Large{Fakultät für Physik}\\
			\end{center}
\begin{verbatim}


\end{verbatim}
							%Eintrag des Wintersemesters
			\begin{center}
			\textbf{\LARGE WS 2013/14}
			\end{center}
\begin{verbatim}


\end{verbatim}
			\begin{center}
			\textbf{\LARGE{Physikalisches Praktikum\\ für das Bachelorstudium}}
			\end{center}
\begin{verbatim}




\end{verbatim}

			\begin{center}
			\textbf{\LARGE{PROTOKOLL}}
			\end{center}
			
\begin{verbatim}

\end{verbatim}

			\begin{flushleft}
			\textbf{\Large{Experiment (Nr., Titel):}}\\
							%Experiment Nr. und Titel statt den Punkten eintragen
			\LARGE{PW2 - Grundgrößen der Mechanik}	
			\end{flushleft}

\begin{verbatim}

\end{verbatim}	
							%Eintragen des Abgabedatums, oder des Erstelldatums des Protokolls
			\begin{flushleft}
			\textbf{\Large{Datum:}} \Large{23.1.2014}
			\end{flushleft}
			
\begin{verbatim}
\end{verbatim}
							%Namen der Protokollschreiber
		\begin{flushleft}
			\textbf{\Large{Namen:}} \Large{Patrick Braun, Johannes Kurz}
			\end{flushleft}

\begin{verbatim}


\end{verbatim}
							%Kurstag und Gruppennummer, zb. Fr/5
			\begin{flushleft}
			\textbf{\Large{Kurstag/Gruppe:}} \Large{DO/2}
			\end{flushleft}

\begin{verbatim}

\end{verbatim}
							%Name des Betreuers, das Praktikum betreute.
			\begin{flushleft}
			\LARGE{\textbf{Betreuer:}}	\Large{Franz Sachslehner}	
			\end{flushleft}

%%%%%%% DECKBLATT ENDE %%%%%%%
\pagebreak
\setlength{\columnsep}{20pt}
\begin{multicols}{2}

%%%%%%%%%%%%%%%%%%%%%%%%%%%%%%%%%%%%%%%%%%%%%%%%

%\begin{figure}[H]
%	\centering
%	\includegraphics[scale=0.35]{./figure/beugung.png}
%	\caption{Beugungsmuster Einzelspalt (echtes Foto; schwarz durch weiß ersetzt)}
%	\label{fig:beugungsmuster}
%\end{figure}


%\begin{figure}[H]
%	\centering
%	\pgfplotstabletypeset[
%			columns={abstand, n},
%			col sep=&,
%			columns/abstand/.style={precision=2, zerofill, column name=\makecell{$Abstand$\\$(\pm 0.05)[mm]$} }, 
%			columns/n/.style={column name=\makecell{$n$\\$(Ordnung)$}, precision=0},
%			every head row/.style={before row=\hline,after row=\hline\hline},
%			every last row/.style={after row=\hline},
%			every first column/.style={column type/.add={|}{} },
%			every last column/.style={column type/.add={}{|} }
%			]{
%			abstand & n
%			12.9 & 1
%			24.45 & 2
%			37.40 & 3
%			49.35& 4
%			62.45 & 5
%			74.45 & 6
%			87.45 & 7
%			100.25 & 8
%			
%			}
%	\caption{Messwerte Einzelspalt}
%	\label{tab:werte_einzelspalt}
%\end{figure}
%

%%%%%%%%%%%%%%%%%%%%%%%%%%%%%%%%%%%%%%%%
\section{Dichte von Flüssigkeiten}

\subsection{Theorie}

$$\rho = \frac{M}{V} = \frac{[kg]}{[m^3]}$$

\subsection{Aufbau und Methoden}

\subsection{Resultate}

\textbf{Pyknometer}
$$m_0 = (31.4 \pm 0.1)g$$
$$m_1 = (82.7 \pm 0.1)g$$
$$m_2 = (84.6 \pm 0.1)g$$

$$T_{dest} = 22.9^\circ \pm 0.1^\circ$$

$$\rho_{dest} = (997.5600 \pm xy) \frac{kg}{m^3}$$

$$\rho_{Probe} = (1034.5067 \pm xy) \frac{kg}{m^3}$$

\textbf{Auftrieb}
$$m_{verdrängt} = (9.973 \pm 0.001)g$$ Wasser\\
$$m_{verdrängt} = (10.347 \pm 0.001)g$$ Kupfersulfat\\

\textbf{Densiometer}
4 Messungen:
$$\rho_{1.1}:   (1.0347 \pm 0.001) \frac{g}{cm^3}$$
$$\rho_{1.2}:   (1.0366 \pm 0.001) \frac{g}{cm^3}$$
Gereinigt:
$$\rho_{2.1}:   (1.0366 \pm 0.001) \frac{g}{cm^3}$$
$$\rho_{2.2}:   (1.0364 \pm 0.001) \frac{g}{cm^3}$$
\subsection{Diskussion}



%%%%%%%%%%%%%%%%%%%%%%%%%%%%%%%%%%%%%%%%
\section{Gleichmäßig beschleunigte Bewegung, eindimensional}

\subsection{Theorie}

g \ldots Gravitationsbeschleunigung Erde\\
m \ldots kleine Masse am Faden\\
M \ldots große Masse am Tisch\\
$\mu$ \ldots Reibungskoeffizient von M\\
a \ldots Beschleunigung beider Massen\\

$$F_g = m * g = [N]$$
$$F_{Reibung} = M*g*\mu$$
$$F_{mM} = (m+M)*a$$
$$F_{g} = F_{mM} + F_{Reibung}$$
$$m * g = M*g*\mu + (m+M)*a$$

\subsection{Aufbau und Methoden}

\subsection{Resultate}
Polynomieller Fit x-ter Ordnung mit der Formel:
$$x(t) = A_0 + B_1*t^1 + B_2 * t^2$$
$$v(t) = B_1 + B_2 * 2 * t$$
$$a(t) = B_2 * 2 = const$$
\end{multicols}
\begin{figure}[H]
	\centering
	\includegraphics[scale=0.4]{./figure/x_t_diagram.png}
	\caption{x(t)-Diagram}
	\label{fig:lin_bewegung}
\end{figure}
\begin{figure}[H]
	\centering
	\includegraphics[scale=0.4]{./figure/v_a_t_diagram.png}
	\caption{v(t)-Diagram und a(t)-Diagram}
	\label{fig:lin_bewegung_v_a}
\end{figure}
\begin{multicols}{2}

\subsection{Diskussion}

\end{multicols}

\begin{figure}[H]
	\centering
	\includegraphics[scale=0.25, angle=-90]{./figure/bewegungen.jpg}
	\caption{Eindimensionale Beschleunigung und kräftefreie Bewegung}
	\label{fig:bewegungen}
\end{figure}

\begin{multicols}{2}
%%%%%%%%%%%%%%%%%%%%%%%%%%%%%%%%%%%%%%%%
\section{Kräftefreie Bewegung}

\subsection{Theorie}

\subsection{Aufbau und Methoden}

\subsection{Resultate}

\subsection{Diskussion}


%%%%%%%%%%%%%%%%%%%%%%%%%%%%%%%%%%%%%%%%
\section{Elastischer Stoß}

\subsection{Theorie}

\subsection{Aufbau und Methoden}

\subsection{Resultate}

\subsection{Diskussion}



%%%%%%%%%%%%%%%%%%%%%%%%%%%%%%%%%%%%%%%%
\section{Inelastischer Stoß}

\subsection{Theorie}

\subsection{Aufbau und Methoden}

\subsection{Resultate}

\subsection{Diskussion}


\section{Quellen}
$[1]$ Anleitung, \url{http://www.univie.ac.at/anfpra/neu1/pw/pw2/PW2.pdf}\\
\end{multicols}

\end{document}