% -*- TeX:de -*-
\NeedsTeXFormat{LaTeX2e}
\documentclass[12pt,a4paper]{article}
\usepackage[german]{babel} % german text
\usepackage[DIV12]{typearea} % size of printable area
\usepackage[T1]{fontenc} % font encoding
%\usepackage[latin1]{inputenc} % most likely on Windows
\usepackage[utf8]{inputenc} % probably on Linux
\usepackage{multicol}

% PLOTTING
\usepackage{pgfplots} 
\usepackage{pgfplotstable}
\usepackage{url}
\usepackage{graphicx} % to include images
\usepackage{tikz}
\usepackage{subfigure} % for creating subfigures
\usepackage{amsmath} % a bunch of symbols
\usepackage{amssymb} % even more symbols
\usepackage{booktabs} % pretty tables

% a floating environment for circuits
\usepackage{float}
\usepackage{caption}

%\newfloat{circuit}{tbph}{circuits}
%\floatname{circuit}{Schaltplan}

% a floating environment for diagrams
%\newfloat{diagram}{tbph}{diagrams}
%\floatname{diagram}{Diagramm}

\selectlanguage{german} % use german

\begin{document}

%%%%%%% DECKBLATT %%%%%%%
\thispagestyle{empty}
			\begin{center}
			\Large{Fakultät für Physik}\\
			\end{center}
\begin{verbatim}


\end{verbatim}
							%Eintrag des Wintersemesters
			\begin{center}
			\textbf{\LARGE WS 2013/14}
			\end{center}
\begin{verbatim}


\end{verbatim}
			\begin{center}
			\textbf{\LARGE{Physikalisches Praktikum\\ für das Bachelorstudium}}
			\end{center}
\begin{verbatim}




\end{verbatim}

			\begin{center}
			\textbf{\LARGE{PROTOKOLL}}
			\end{center}
			
\begin{verbatim}





\end{verbatim}

			\begin{flushleft}
			\textbf{\Large{Experiment (Nr., Titel):}}\\
							%Experiment Nr. und Titel statt den Punkten eintragen
			\LARGE{PW4 Oberflächenspannung, Viskosität, Hygrometrie, Schmelzwärme}	
			\end{flushleft}

\begin{verbatim}

\end{verbatim}	
							%Eintragen des Abgabedatums, oder des Erstelldatums des Protokolls
			\begin{flushleft}
			\textbf{\Large{Datum:}} \Large{31.10.2013}
			\end{flushleft}
			
\begin{verbatim}
\end{verbatim}
							%Namen der Protokollschreiber
		\begin{flushleft}
			\textbf{\Large{Namen:}} \Large{Patrick Braun, Johannes Kurz}
			\end{flushleft}

\begin{verbatim}


\end{verbatim}
							%Kurstag und Gruppennummer, zb. Fr/5
			\begin{flushleft}
			\textbf{\Large{Kurstag/Gruppe:}} \Large{DO/2}
			\end{flushleft}

\begin{verbatim}



\end{verbatim}
							%Name des Betreuers, das Praktikum betreute.
			\begin{flushleft}
			\LARGE{\textbf{Betreuer:}}	\Large{ Franz Sachslehner }	
			\end{flushleft}

%%%%%%% DECKBLATT ENDE %%%%%%%
\pagebreak
\setlength{\columnsep}{20pt}
\begin{multicols}{2}
\section{Einleitung}

\section{Oberflächenspannung}
Ziel des Experiments ist es die Oberflächenspannung einer Flüssigkeit zu bestimmen. Dazu wird ein Metallring an einer Waage befestigt und mit einer Skala und Gewichten geeicht. Taucht man den Ring in die Flüssigkeit und zieht den Flüssigkeitsbehälter langsam nach unten, kann beim Reißen der sich bildenden Flüssigkeitsmembran an der Skala die Kraft abgelesen werden. Den Aufbau und die Eichgewichte sind in Abbildung \ref{fig:oberflaeche_eichung_aufbau} ersichtlich.
\begin{figure}[H]
	\centering
	\includegraphics[scale=0.2]{./figure/Waageneichung-Aufbau.png}
	\caption{Foto \\Versuchsaufbau und Eichung}
	\label{fig:oberflaeche_eichung_aufbau}
\end{figure}
\noindent
Bevor Messungen durchgeführt werden können muss wie erwähnt die verwendete Waage geeicht werden. Dazu werden auf die Schale welche zwischen Ring und Befestigung angebracht ist, 0.1g schwere Metallstücke gelegt und die dazugehörigen Striche abgelesen. Dies wird von 0.1g bis 1.4g durchgeführt. Dadurch kann die Kraft pro Strich mit
$$K = m * a$$
errechnet werden. K ist ein empirischer wert den wir bei der Messung von der Skala ablesen.
Weiters ist die Oberflächenspannung abhängig von der Fläche des Ringes. Das Verhältnis zwischen Zugkraft und Fläche ergibt die Oberflächenspannung:
$$\sigma = \frac{K}{\pi * (d_1 + d_2)}$$
\noindent
Folgende Materialien wurden verwendet:
\begin{itemize}
	\item Eichgewichter
	\item Eine Pinzette
	\item Eine Waage
	\item Schale mit Wasser
	\item Metallring mit Aufhängung
	\item Ein beweglicher Objekttisch
\end{itemize}
Bei der Durchführung wurde zu erst normals- und um danach destilliertes Wasser verwendet um einen Vergleich zu ermöglichen. Näheres dazu wird im Punkt Diskussion erläutert.
\subsection{Messwerte und Ergebnisse}
Die Eichung der Waage ergab einen linearen Anstieg (Abbildung \ref{fig:oberflaeche_eichung_fit}) über den wir die Kraft pro Strich im Mittel genau errechnen können.
\begin{figure}[H]
	\centering
	\includegraphics[scale=0.45]{./figure/Waageneichung-Fit.png}
	\caption{Skizze Versuchsaufbau}
	\label{fig:oberflaeche_eichung_fit}
\end{figure}
\noindent
% Regress in N gegen Strich
%%%   B (y-intercept) = 6,373626373626351e-02 +/- 2,122709832588890e-02
%%%   A (slope) = 4,651006485868871e+02 +/- 2,541280599100647e+00
Durch die lineare Regression (Abbildung \ref{fig:oberflaeche_eichung_linreg}) der Messwerte (Tabelle \ref{fig:oberflaeche_eichung}) erhält man die Steigung der Kraft pro strich. Diese Steigung entspricht pro Strich \textbf{$K = \frac{1}{465} = (0.0022\pm )N$}. Für den Metallring wurde mit einer Schiebelehre wie folgt gemessen:\\
Außendurchmesser: $(24.00 \pm 0.05)mm$\\
Innendurchmesser: $(23.15 \pm 0.05)mm$\\
Errechnet man sich die Oberflächenspannung zusammen mit den Werten aus Tabelle \ref{fig:oberflaeche_messung} für $Wasser_{destilliert}$ ergibt sich:
$$\sigma = \frac{4.62 * 0.0022N}{\pi * (0.02400 + 0.02315)m} =$$ 
$$= (0.069 \pm 0.00) \frac{N}{m}$$ %TODO Fehler
Für Wasser mit Verunreinigungen: 
$$ \sigma = \frac{3.9 * 0.0022N}{\pi * (0.02400 + 0.02315)m} =$$
$$ = (0.058 \pm 0.000) \frac{N}{m}$$

\begin{figure}[H]
	\centering
	\pgfplotstabletypeset[
			columns={g,strich},
			col sep=tab,
			columns/T1/.style={column name=$T_1[s]$},
			columns/T2/.style={column name=$T_2[s]$},
			every head row/.style={before row=\hline,after row=\hline\hline},
			every last row/.style={after row=\hline},
			every first column/.style={
								column type/.add={|}{}
							        },
			every last column/.style={
								column type/.add={}{|}
								}
			]{./data/oberflaechenspannung_eichung.dat}
	\caption{Eichungsmessung der Waage}
	\label{fig:oberflaeche_eichung}
\end{figure}
\noindent

\begin{figure}[H]
	\centering
	\pgfplotstabletypeset[
			col sep=space,
			every head row/.style={before row=\hline,after row=\hline\hline},
			every last row/.style={after row=\hline},
			every first column/.style={
								column type/.add={|}{}
							        },
			every last column/.style={
								column type/.add={}{|}
								}
			]{
			Wasser $Wasser_{destilliert}$
			3.9 4.7
			3.8 4.6
			3.9 4.6
			3.9 4.6
			3.9 4.6
			3.9 0
			4.0 0
			}
	\caption{Messung von Wasser und destilliertem Wasser}
	\label{fig:oberflaeche_messung}
\end{figure}
\noindent
\\
Temperatur erste Messung:\\
$23.5^{\circ}C$\\
Temperatur zweite Messung:\\
$22.3^{\circ}C$\\
\subsection{Diskussion}
Die Durchführung der Eichung und Messung gestaltete sich unproblematisch. Die Eichung erfordert etwas Fingerspitzengefühl und Geduld da die Waage einen (fast) umgedämpften Oszillator gleicht. Bei den ersten Zwischenrechnungen ergab sich ein zu niedriger Wert im Vergleich zum Literaturwert. \\
Literaturwert: \\
Wasser bei $20^{\circ}$ $\sigma = 72.75 \frac{mN}{m}$.\\
Daraufhin wurde das Wasser in der Schale nach der Reinigung derselben durch destilliertes Wasser ersetzt um die Verunreinigungen so niedrig wie möglich zu halten. Zusätzlich war das Wasser um ca. $1^{\circ}C$ kühler was ebenfalls zu einer Näherung an den Literaturwert geführt hat. Mit einem Wert von $(69 \pm 0.00) \frac{mN}{m}$ liegt die Messung bei $22.3^{\circ}C$ bei einem zu erwartenden Wert. Die Differenz ist hauptsächlich durch den Temperaturunterschied zu erklären und an den Messungenauigkeiten der verwendeten Geräte. %TODO Fehler

%%%%%%%%%%%%%%%%%%%%%%%%%%%%%%%%%%%%
\section{Viskosität}
Rühre, Pinzette, Kugeln
Thermometer Sumit DT150\\
\subsection{Messwerte und Ergebnisse}
Temp: $24^{\circ}C \pm 1^{\circ}C$\\
Länge Innen-Innen: 13.42cm\\
Dichte Flüssigkeit: $1225 \pm 5 kg/m^2$\\
\\
Messungen 1:\\
$0.794 \pm 0.002mm$\\
$16.1 \pm 0.1 mg$\\

Fallzeiten:\\
2.4s\\
2.37s\\
2.4s\\
2.38\\
2.41\\
2.41\\
2.31\\
2.44\\
2.47\\
2.44\\
2.47\\
\\
Messung 2:\\
$6.6 \pm 0.1mg$\\
$0.595 \pm 0.01mm$\\
\\
Fallzeiten:\\
3.94\\
3.97\\
3.97\\
4.13\\
3.97\\
3.97\\
3.97\\
3.79\\
3.97\\
3.88\\
4.04\\
\\
\textbf{Höppler-Viskosimeter}\\
HAAKE, Thermometer, Heizgerät\\
Temperatur: $25.0^{\circ}C \pm 0.1^{\circ}C$\\
Messung 1: 2m51s90ms\\
Messung 2: 2m51s32ms\\
Messung 3: 2m51s32ms\\
\\
Temperatur: $50.6^{\circ}C \pm 0.1^{\circ}C$\\
Messung 1: 0m42s25ms\\
Messung 2: 0m41s63ms\\
Messung 3: 0m41s78ms\\
\\
\subsection{Diskussion}


%%%%%%%%%%%%%%%%%%%%%%%%%%%%%%%%%%%%
\section{Luftfeuchtigkeit}
\subsection{Messwerte und Ergebnisse}
$T_{Start} = 24^{\circ}C \pm 1^{\circ}C$\\

\subsection{Diskussion}

\section{Schmelztemperatur von Wasser}

\subsection{Messwerte und Ergebnisse}

\subsection{Diskussion}

\section{Quellen}
Formeln und Abbildung Fahrradpendel:\\
\url{http://www.univie.ac.at/anfpra/neu1/pw/pw3/PW3.pdf}\\


\end{multicols}
\end{document}