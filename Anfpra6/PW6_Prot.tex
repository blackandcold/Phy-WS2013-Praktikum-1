% -*- TeX:de -*-
\NeedsTeXFormat{LaTeX2e}
\documentclass[12pt,a4paper]{article}
\usepackage[german]{babel} % german text
\usepackage[DIV12]{typearea} % size of printable area
\usepackage[T1]{fontenc} % font encoding
%\usepackage[latin1]{inputenc} % most likely on Windows
\usepackage[utf8]{inputenc} % probably on Linux
\usepackage{multicol}

% PLOTTING
\usepackage{pgfplots} 
\usepackage{pgfplotstable}
\usepackage{url}
\usepackage{graphicx} % to include images
\usepackage{tikz}
\usepackage{subfigure} % for creating subfigures
\usepackage{amsmath} % a bunch of symbols
\usepackage{amssymb} % even more symbols
\usepackage{booktabs} % pretty tables
\usepackage{makecell} % multi row table heading

% a floating environment for circuits
\usepackage{float}
\usepackage{caption}

%\newfloat{circuit}{tbph}{circuits}
%\floatname{circuit}{Schaltplan}

% a floating environment for diagrams
%\newfloat{diagram}{tbph}{diagrams}
%\floatname{diagram}{Diagramm}

\selectlanguage{german} % use german

\begin{document}








%%%% TO DO
%
% - - Shorty:
%
% - - Tabelle "Messwerte Linsenbrennweite"
%		bitte bei jedem neuen e eine trennlinie... bin zu deppert ^^

% - - Patrick
%




%%%%%%% DECKBLATT %%%%%%%
\thispagestyle{empty}
			\begin{center}
			\Large{Fakultät für Physik}\\
			\end{center}
\begin{verbatim}


\end{verbatim}
							%Eintrag des Wintersemesters
			\begin{center}
			\textbf{\LARGE WS 2013/14}
			\end{center}
\begin{verbatim}


\end{verbatim}
			\begin{center}
			\textbf{\LARGE{Physikalisches Praktikum\\ für das Bachelorstudium}}
			\end{center}
\begin{verbatim}




\end{verbatim}

			\begin{center}
			\textbf{\LARGE{PROTOKOLL}}
			\end{center}
			
\begin{verbatim}





\end{verbatim}

			\begin{flushleft}
			\textbf{\Large{Experiment (Nr., Titel):}}\\
							%Experiment Nr. und Titel statt den Punkten eintragen
			\LARGE{PW6 Geometrische Optik}	
			\end{flushleft}

\begin{verbatim}

\end{verbatim}	
							%Eintragen des Abgabedatums, oder des Erstelldatums des Protokolls
			\begin{flushleft}
			\textbf{\Large{Datum:}} \Large{14.11.2013}
			\end{flushleft}
			
\begin{verbatim}
\end{verbatim}
							%Namen der Protokollschreiber
		\begin{flushleft}
			\textbf{\Large{Namen:}} \Large{Patrick Braun, Johannes Kurz}
			\end{flushleft}

\begin{verbatim}


\end{verbatim}
							%Kurstag und Gruppennummer, zb. Fr/5
			\begin{flushleft}
			\textbf{\Large{Kurstag/Gruppe:}} \Large{DO/2}
			\end{flushleft}

\begin{verbatim}



\end{verbatim}
							%Name des Betreuers, das Praktikum betreute.
			\begin{flushleft}
			\LARGE{\textbf{Betreuer:}}	\Large{ Johanna Akbarzadeh }	
			\end{flushleft}

%%%%%%% DECKBLATT ENDE %%%%%%%
\pagebreak
\setlength{\columnsep}{20pt}
\begin{multicols}{2}

%%%%%%%%%%%%%%%%%%%%%%%%%%%%%%%%%%%%
\section{Brennweite von Linsen}
Die Brennweite $f $ einer Linse ist der Abstand zwischen Brennpunkt und der angenommenen Brechungsebene einer Linse. \\
Für Sammellinsen (konvexe) gehen (idealerweise) alle parallel zur optischen Achse einfallenden Strahlen durch den Brennpunkt.\\
Für Konkave Linsen, die Parallelstrahlen so brechen, dass sie auseinanderstreuen, liegt ein imaginärer Brennpunkt vor der Linse, in welchem die Verlängerungen der Strahlen gebündelt sind. Daher haben die Brennweiten von Zerstreuungslinsen ein negatives Vorzeichen.\\
\\
In diesem Beispiel soll die Brennweite von einer Konkav- und einer Konvexlinse bestimmt werden.\\
Durch geometrische Überlegungen, aus der Abbildung eines Gegenstands durch eine Linse (Gegenstandsweite $g$, Bildweite $b$) lässt sich, anhand der 3 Hauptstrahlen die Abbildungsgleichung für Linsen herleiten:
$$\frac{1}{f}=\frac{1}{g}+ \frac{1}{b}$$
Zur Bestimmung wird ein Gegenstand (hier eine Schablone) durch eine Linse auf einen Abbildungsschirm scharf gestellt. Durch mehrmalige Messung der Gegenstands- und Bildweite, wird also die Brennweite einer Konvexlinse bestimmt.\\
Eine zweite in diesem Versuch angewandte Methode ist das Besselverfahren:\\
Es wird dabei ausgenutzt, dass bei festem Abstand $e$ zwischen Gegenstand und Bild, an 2 Positionen der Linse ein scharfe Abbildung gegeben ist (jedenfalls $e > 4\cdot f$).\\
$d =$ $Abstand$ $zwischen$ $beiden$ $Positionen$
$e=$ $Abstand$ $Gegenstand-Schirm$\\
\\
Aus der Beziehung $e = g + b$ und $d=|g-b|$ lässt sich durch Umformungen aus der Abbildungsgleichung erhalten:
$$f=\frac{1}{4} \left(e-\frac{d^2}{e} \right)$$
\\
Zuletzt soll die Brennweite einer Konkavlinse bestimmt werden.\\
Da diese kein reelles Bild liefert, wird sie zur Messung gemeinsam mit der zuvor vermessenen Konvexlinse auf den Lichtstrahl angewendet. Nachdem das Bild scharf gestellt wurde, lässt sich die Bildweite, jedoch nicht die Gegenstandsweite direkt messen. Diese ergibt sich (in Kenntnis der zuvor gemessenen Bildweite der konvexen) durch
$$g_{konkav}= -(b_{konvex}-d_{konv-konk})$$


\subsection{Messwerte und Ergebnisse}



\begin{figure}[H]
	\centering
	\pgfplotstabletypeset[
			columns={e,g1,g2},
			col sep=&,
			columns/e/.style={column name=\makecell{$e$\\$[cm]$}, precision=1, zerofill}, %
			columns/g1/.style={column name=\makecell{$g1$\\$[cm]$}, precision=1, zerofill},
			columns/g2/.style={column name=\makecell{$g2$\\$[cm]$}, precision=1, zerofill},
			every head row/.style={before row=\hline,after row=\hline\hline},
			every last row/.style={after row=\hline},
			every first column/.style={
								column type/.add={|}{}
							        },
			every last column/.style={
								column type/.add={}{|}
								}
			]{
			e & g1 & g2
			40 & 17.5 & 22.8
			      & 17.3 & 22.5
			      & 17.2 & 23.5
			      & 17.5 & 23.5
			      & 17.5 & 23.0
			
			50 & 13.4 & 36.8
			      & 13.3 & 36.7
			45 & 14.5 & 30.6
			      & 14.5 & 30.7
			60 &  12.4 & 47.8
			65 & 12.0 & 53.0
			}
	\caption{Je 2 Gegenstandsweiten bei verschiedenen Schirmabständen - Konvexlinse}
	\label{fig:Gegenstandsweiten_Konvexlinse}
\end{figure}

\begin{itemize}
	\item $e = (40.0 \pm 0.1) cm$\\
	$g_1 = (17.4 \pm 0.1)cm$\\
	$g_2 =(23.1 \pm 0.2) cm$\\
	\\
	$$f_{G/B}=(9.831\pm 0.023) cm$$
	$$\widehat{=}(0.10172 \pm 0.00024)Dioptrien$$
	$$f_{Bessel} = (9.797 \pm 0.031)cm$$
	
	
	\item $e = (50.0 \pm 0.1)cm$\\
	\indent $g_1 = (13.4 \pm 0.2) cm$\\
	$g_2 = (36.8 \pm 0.2)cm$
	$$f_{Bessel} = (9.762 \pm 0.099) cm$$
	
	\item $e= (45.0 \pm 0.1)cm$\\
	\indent $g_1 = (14.5 \pm 0.2) cm$\\
	$g_2 = (30.6 \pm 0.2) cm$
	$$f_{Bessel}=(9.810 \pm 0.049cm)$$
	
	\item $e=(60.0 \pm 0.1)cm$\\
	\indent $g_1= (12.4 \pm 0.2) cm$\\
	$g_2 = (47.8 \pm 0.2)cm$
	$$f_{Bessel}=(9.78 \pm 0.13) cm$$
	
	\item $e=(65.0 \pm 0.1)cm$\\
	\indent $g_1 = (12.0 \pm 0.2) cm$\\
	$g_2 =(53.0 \pm 0.2)cm$
	$$f_{Bessel} = (9.78 \pm 0.14)cm$$

\end{itemize}

\textbf{Mittel aller 5 Besselmessungen:}\\
$$f=(9.786 \pm 0.045)$$
$$Brechkraft: (0.10219 \pm 0.00047) Dioptrien$$\\
\\
\textbf{Konkavlinse:}
\begin{itemize}
	\item $b_{konvex}=(14.4 \pm 0.2) cm$\\
	$b_{konkav}=(16.7 \pm 0.2)cm$\\
	$d=(8.4 \pm 0.1)cm$
	$$f_{konkav}= (-9.36 \pm 0.70)cm$$
	$$\widehat{=}(-0.10684 \pm 0.00052)Dioptrien$$
	
	\item $b_{konvex}=(17.2 \pm 0.2)cm$\\
	$b_{konvex}=(59.2 \pm 0.4)cm$\\
	$d=(9.2 \pm 0.1)cm$
	$$f_{konkav}= (-9.25 \pm 0.30)cm$$
	$$\widehat{=}(-0.10811 \pm 0.00036)Dioptrien$$
	
	
\end{itemize}




\subsection{Diskussion}
Aus der Messreihe der Gegenstandsweite für $e=40.0cm$ wurde der erste gemessene Wert für $g_1$ entfernt. Dieser wurde erhalten, noch bevor das Licht im Raum gedimmt wurde und eine geeignete Arbeitsweise zum Scharfstellen der Abbildungen gefunden wurde. Daher darf der Messwert, der mit $g_1 = 16.7 cm$ deutlich niedriger als alle 5 anderen ist, als falsch angesehen werden.\\
\\
An den Ergebnissen für die Brennweite $f$ der Konvexlinse gibt es vor allem 2 Auffälligkeiten:\\
- Die verschieden Ergebnisse aus dem Besselverfahren liegen knapp und plausibel beieinander, das Ergebnis aus der Abbildungsgleichung liegt jedoch, wenn auch nur leicht, außerhalb des Unsicherheitsbereiches des Bessel-Gesamtergebnisses.\\
- Sowohl für die konvexe als auch die konkave Linse liegt der Herstellerwert für die Brennweite bei ($\pm$) 10cm.\\
\\
Bei der (stichprobenartigen) Berechnung von $f_{konvex}$ durch rückrechnen auf die Bildweite aus $e$ und $g$ der anderen Daten aus dem Besselverfahren, zeigt sich jedenfalls eine etwas breitere Streuung:\\
$f = 9.715cm$ für $e=50cm$, $g=36.8cm$\\
$f=9.809cm$ für $e=50cm$, $g=13.4cm$\\
oder auch:\\
$f=9.828cm$ für $e=45cm$, $g=14.5cm$\\
$f=9.837cm$ für $e=60cm$, $g=12.4cm$\\
\\
Da der Bereich der Abweichung dennoch relativ klein ist, lässt sich aus den vorliegenden Daten, diesem Befund nicht genauer auf den Grund gehen.\\
Um einen systematischen Fehler zu finden, oder dem Besselverfahren höhere Genauigkeit zuschreiben zu können, wäre es nötig, mehrere Linse zu überprüfen und eine ausreichend große Sichprobenanzahl anzufertigen, wo auch für eine gleichmäßige Menge an Einzelmessungen gesorgt wurde. Dies war hier nicht der Fall, da das Ziel die Bestimmung der Brennweite einer Linse und nicht der Vergleich der Methoden war.\\
\\
Die Abweichung der Brennweiten vom angegebenen Wert, die für alle Ergebnisse in eine Richtung geht und für die konvexe Linse im Ausmaß konstant ist, lässt 2 triviale Schlüsse zu: \\
Die Herstellerangabe ist entweder ungenau, oder die Messung fehlerbehaftet.\\
Da alle Ergebnisse nah beieinander liegen und für beide Linsen der Betrag der Brennweite kleiner, als erwartet, gemessen wurde, liegt sehr wahrscheinlich ein systematischer Fehler vor. \\
Eine plausible Möglichkeit wäre, dass der Gegenstand zu groß ist, im Verhältnis zur Linse und dadurch eine kleinere Brennweite durch sphärische Aberration (siehe Kap.2 - Linsenfehler) gemessen wurde.\\
In Kenntnis des Unsicherheitsbereiches der Herstellerangabe und, indem der Versuch mit verschiedenen Schablonen (und vielleicht auch verschiedenen Lampen und Schirmen) durchgeführt würde, ließe sich dem auf den Grund gehen.


%%%%%%%%%%%%%%%%%%%%%%%%%%%%%%%%%%%%
\section{Linsenfehler}

\subsection{Messwerte und Ergebnisse}

\subsection{Diskussion}

%%%%%%%%%%%%%%%%%%%%%%%%%%%%%%%%%%%%
\section{Microskop}

\subsection{Messwerte und Ergebnisse}

\subsection{Diskussion}

%%%%%%%%%%%%%%%%%%%%%%%%%%%%%%%%%%%%
\section{Fernrohr}

\subsection{Messwerte und Ergebnisse}

\subsection{Diskussion}

\section{Quellen}
$[1]$ Leitfaden, \url{http://www.univie.ac.at/anfpra/neu1/pw/pw6/PW6PL4.pdf}\\
%$[2]$ 

\end{multicols}
\end{document}