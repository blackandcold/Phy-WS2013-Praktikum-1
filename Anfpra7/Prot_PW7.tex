% -*- TeX:de -*-
\NeedsTeXFormat{LaTeX2e}
\documentclass[12pt,a4paper]{article}
\usepackage[german]{babel} % german text
\usepackage[DIV12]{typearea} % size of printable area
\usepackage[T1]{fontenc} % font encoding
%\usepackage[latin1]{inputenc} % most likely on Windows
\usepackage[utf8]{inputenc} % probably on Linux
\usepackage{multicol}

% PLOTTING
\usepackage{pgfplots} 
\usepackage{pgfplotstable}
\usepackage{url}
\usepackage{graphicx} % to include images
\usepackage{tikz}
\usepackage{subfigure} % for creating subfigures
\usepackage{amsmath} % a bunch of symbols
\usepackage{amssymb} % even more symbols
\usepackage{booktabs} % pretty tables
\usepackage{makecell} % multi row table heading

% a floating environment for circuits
\usepackage{float}
\usepackage{caption}

%\newfloat{circuit}{tbph}{circuits}
%\floatname{circuit}{Schaltplan}

% a floating environment for diagrams
%\newfloat{diagram}{tbph}{diagrams}
%\floatname{diagram}{Diagramm}

\selectlanguage{german} % use german

\begin{document}








%%%% TO DO
%
% - - Shorty:
%
% - - Tabelle "Messwerte Linsenbrennweite"
%		bitte bei jedem neuen e eine trennlinie... bin zu deppert ^^

% - - Patrick
%




%%%%%%% DECKBLATT %%%%%%%
\thispagestyle{empty}
			\begin{center}
			\Large{Fakultät für Physik}\\
			\end{center}
\begin{verbatim}


\end{verbatim}
							%Eintrag des Wintersemesters
			\begin{center}
			\textbf{\LARGE WS 2013/14}
			\end{center}
\begin{verbatim}


\end{verbatim}
			\begin{center}
			\textbf{\LARGE{Physikalisches Praktikum\\ für das Bachelorstudium}}
			\end{center}
\begin{verbatim}




\end{verbatim}

			\begin{center}
			\textbf{\LARGE{PROTOKOLL}}
			\end{center}
			
\begin{verbatim}





\end{verbatim}

			\begin{flushleft}
			\textbf{\Large{Experiment (Nr., Titel):}}\\
							%Experiment Nr. und Titel statt den Punkten eintragen
			\LARGE{PW7 Brechung, Dispersion, Refraktometrie}	
			\end{flushleft}

\begin{verbatim}

\end{verbatim}	
							%Eintragen des Abgabedatums, oder des Erstelldatums des Protokolls
			\begin{flushleft}
			\textbf{\Large{Datum:}} \Large{21.11.2013}
			\end{flushleft}
			
\begin{verbatim}
\end{verbatim}
							%Namen der Protokollschreiber
		\begin{flushleft}
			\textbf{\Large{Namen:}} \Large{Patrick Braun, Johannes Kurz}
			\end{flushleft}

\begin{verbatim}


\end{verbatim}
							%Kurstag und Gruppennummer, zb. Fr/5
			\begin{flushleft}
			\textbf{\Large{Kurstag/Gruppe:}} \Large{DO/2}
			\end{flushleft}

\begin{verbatim}



\end{verbatim}
							%Name des Betreuers, das Praktikum betreute.
			\begin{flushleft}
			\LARGE{\textbf{Betreuer:}}	\Large{ Johanna Akbarzadeh }	
			\end{flushleft}

%%%%%%% DECKBLATT ENDE %%%%%%%
\pagebreak
\setlength{\columnsep}{20pt}
\begin{multicols}{2}


%\begin{figure}[H]
%	\centering
%	\includegraphics[scale=0.13]{./figure/linsenfehler.jpg}
%	\caption{Versuchaufbau Konvex-Plan Linse}
%	\label{fig:linsenfehler_aufbau}
%\end{figure}
%
%Glasprisma gegen Luft
%1.000292/sin(39grad28minutes)

%%%%%%%%%%%%%%%%%%%%%%%%%%%%%%%%%%%%%%%%%%%%%%%%
\section{Dispersionskurve eines Glasprismas}
Cadmiumlampe, 5 Linien, 
Im ersten Experiment soll ein Glasprisma untersucht werden. Dabei wird eine Cadmium-Dampflampe verwendet welche 5 Spektren aussendet. Das Licht dieser Lampe wird durch einen Schlitz geregelt und auf das Prisma fokussiert. Zusätzlich ist ein Fernrohr angebracht welches mit einer Winkel-Skala rotierbar ist. Durch Einstellung des Nullpunktes ohne Prisma und anschließendes rotieren mit Prisma des Fernrohres bis zum Punkt der minimalen Ablenkung (Umkehrpunkt der Spektrallinien) kann $\delta_{min}$ für alle Spektren bestimmt werden. Der Versuchsaufbau mit Lampe, Blende, Prisma auf rotierbarem Tisch und Fernrohr ist in Abbildung \ref{fig:prisma_aufbau} ersichtlich. Zur Messung der Wellenlänge wurde ein mit einem Glasfaser verbundener Sensor verwendet welcher sich auf einem höhenverstellbarem Tisch befindet. Ersichtlich am rechten Bildrand. Um diese Messung durch zu führen wurde das Fernrohr weggeschwenkt und durch das Glasfaserkabel ersetzt. Das Prisma ist zu diesem Zweck nicht mehr notwendig.
\\ 
Zusätzlich kann durch die Wellenlänge und dem Winkel der Brechungsindex errechnet werden. Mit diesen Daten kann eine Dispersionskurve gefittet werden.\\
Die Formel für die Kurve:
$$n(\lambda) = A + \frac{B}{\lambda^2}$$

\begin{figure}[H]
	\centering
	\includegraphics[angle=-90,scale=0.11]{./figure/prisma_aufbau.jpg}
	\caption{Versuchsaufbau Prisma}
	\label{fig:prisma_aufbau}
\end{figure}

\subsection{Messwerte und Ergebnisse}

\begin{figure}[H]
	\centering
	\pgfplotstabletypeset[
			columns={w,lambda},
			col sep=&,
			columns/w/.style={string type, column name=\makecell{$Winkel$\\$[Grad]$} }, 
			columns/lambda/.style={column name=\makecell{$Wellenlänge$\\$[nm]$}, precision=0},
			every head row/.style={before row=\hline,after row=\hline\hline},
			every last row/.style={after row=\hline},
			every first column/.style={column type/.add={|}{} },
			every last column/.style={column type/.add={}{|} }
			]{
			w & lambda
			47$^\circ$ 50' & 550
			48$^\circ$ 58' & 508
			49$^\circ$ 29' & 481
			49$^\circ$ 45' & 466
			50$^\circ$ 18' & 441
			
			}
	\caption{Messwerte Winkel und Wellenlänge Cadmium Dampflampe}
	\label{fig:werte_cadmiumdampflampe}
\end{figure}
441 - Dieser Wert konnte nicht aufgelöst werden und ist ein Literaturwert!

% TODO KURVE UND FORMEL ZUR RECHNUNG, Fehler

%A = 1,589733683934140e+00 +/- 9,033659153860940e-04
%B = 9,968439199611275e+03 +/- 2,159644777247912e+02

\end{multicols}
\begin{figure}[H]
	\centering
	\includegraphics[scale=0.45]{./figure/Spektrum_Cadmiumdampflampe_PW7_Braun_Kurz.png}
	\caption{Spektrum Cadmiumlampe - Wellenlänge/Intensität}
	\label{fig:prisma_wellenlaenge}
\end{figure}

\begin{figure}[H]
	\centering
	\includegraphics[scale=0.5]{./figure/anfpra7_spektrum_grafik.png}
	\caption{Brechungsindex abhängig von der Wellenlänge}
	\label{fig:wellenlaenge_brechungsindex}
\end{figure}

\begin{multicols}{2}

\subsection{Diskussion}
Dank neuem Equipment und leichter Bedienbarkeit war der Aufbau kein Problem. Zu beachten ist eine genaue Nullpunktjustierung und die Mitte der Spektrallinien mit dem eingelassenen Fadenkreuz zu treffen um einen möglichst genauen Winkel messen zu können. Dank dem Nonimus ist eine Minutengenaue Messung möglich. Durch die Justierungsmaßnahmen und das Treffen der Mitte einer Spektrallinie entsteht jedoch ein geschätzter Fehler von $\pm 5'$\\
\\
Bei der Messung der Wellenlängen konnte der Sensor und die Software schnell eingerichtet und leicht verwendet werden. Zu bemerken war das Umgebungslicht zu großen Störungen führt und das violette Licht keinen erkennbaren Ausschlag produziert hat. Durch Abdunklung des Raumes konnten die Wellenlängen gemessen werden (\ref{fig:prisma_wellenlaenge}).\\
\\
In Abbildung \ref{fig:wellenlaenge_brechungsindex} ist das verhalten des Brechungsindex zur Wellenlänge aufgetragen. Die Dispersionskurve ist nicht linear!

%%%%%%%%%%%%%%%%%%%%%%%%%%%%%%%%%%%%%%%%%%%%%%%%
\section{Lichtbrechung an einer planparallelen Platte}

\subsection{Messwerte und Ergebnisse}


\subsection{Diskussion}


%%%%%%%%%%%%%%%%%%%%%%%%%%%%%%%%%%%%%%%%%%%%%%%%
\section{Refraktometrie}

\begin{figure}[H]
	\centering
	\includegraphics[angle=-90,scale=0.11]{./figure/refrakto.jpg}
	\caption{Verwendete Refraktometer}
	\label{fig:geraete_refrakto}
\end{figure}

\subsection{Messwerte und Ergebnisse}


\subsection{Diskussion}


\section{Quellen}
$[1]$ Leitfaden, \url{http://www.univie.ac.at/anfpra/neu1/pw/pw7/PW7.pdf}\\

\end{multicols}
\end{document}