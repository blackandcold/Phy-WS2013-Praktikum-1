% -*- TeX:de -*-
\NeedsTeXFormat{LaTeX2e}
\documentclass[12pt,a4paper]{article}
\usepackage[german]{babel} % german text
\usepackage[DIV12]{typearea} % size of printable area
\usepackage[T1]{fontenc} % font encoding
%\usepackage[latin1]{inputenc} % most likely on Windows
\usepackage[utf8]{inputenc} % probably on Linux
\usepackage{multicol}

% PLOTTING
\usepackage{pgfplots} 
\usepackage{pgfplotstable}
\usepackage{url}
\usepackage{graphicx} % to include images
\usepackage{tikz}
\usepackage{subfigure} % for creating subfigures
\usepackage{amsmath} % a bunch of symbols
\usepackage{amssymb} % even more symbols
\usepackage{booktabs} % pretty tables
\usepackage{makecell} % multi row table heading

% a floating environment for circuits
\usepackage{float}
\usepackage{caption}

%\newfloat{circuit}{tbph}{circuits}
%\floatname{circuit}{Schaltplan}

% a floating environment for diagrams
%\newfloat{diagram}{tbph}{diagrams}
%\floatname{diagram}{Diagramm}

\selectlanguage{german} % use german

\begin{document}

%%%%%%% DECKBLATT %%%%%%%
\thispagestyle{empty}
			\begin{center}
			\Large{Fakultät für Physik}\\
			\end{center}
\begin{verbatim}


\end{verbatim}
							%Eintrag des Wintersemesters
			\begin{center}
			\textbf{\LARGE WS 2013/14}
			\end{center}
\begin{verbatim}


\end{verbatim}
			\begin{center}
			\textbf{\LARGE{Physikalisches Praktikum\\ für das Bachelorstudium}}
			\end{center}
\begin{verbatim}




\end{verbatim}

			\begin{center}
			\textbf{\LARGE{PROTOKOLL}}
			\end{center}
			
\begin{verbatim}

\end{verbatim}

			\begin{flushleft}
			\textbf{\Large{Experiment (Nr., Titel):}}\\
							%Experiment Nr. und Titel statt den Punkten eintragen
			\LARGE{PW9 Gleichstrom}	
			\end{flushleft}

\begin{verbatim}

\end{verbatim}	
							%Eintragen des Abgabedatums, oder des Erstelldatums des Protokolls
			\begin{flushleft}
			\textbf{\Large{Datum:}} \Large{21.11.2013}
			\end{flushleft}
			
\begin{verbatim}
\end{verbatim}
							%Namen der Protokollschreiber
		\begin{flushleft}
			\textbf{\Large{Namen:}} \Large{Patrick Braun, Johannes Kurz}
			\end{flushleft}

\begin{verbatim}


\end{verbatim}
							%Kurstag und Gruppennummer, zb. Fr/5
			\begin{flushleft}
			\textbf{\Large{Kurstag/Gruppe:}} \Large{DO/2}
			\end{flushleft}

\begin{verbatim}

\end{verbatim}
							%Name des Betreuers, das Praktikum betreute.
			\begin{flushleft}
			\LARGE{\textbf{Betreuer:}}	\Large{ Clemens Nagel }	
			\end{flushleft}

%%%%%%% DECKBLATT ENDE %%%%%%%
\pagebreak
\setlength{\columnsep}{20pt}
\begin{multicols}{2}

%%%%%%%%%%%%%%%%%%%%%%%%%%%%%%%%%%%%%%%%%%%%%%%%
%\end{multicols}
%\begin{figure}[H]
%	\centering
%	\includegraphics[scale=0.35]{./figure/beugung.png}
%	\caption{Beugungsmuster Einzelspalt (echtes Foto; schwarz durch weiß ersetzt)}
%	\label{fig:beugungsmuster}
%\end{figure}


%\begin{figure}[H]
%	\centering
%	\pgfplotstabletypeset[
%			columns={abstand, n},
%			col sep=&,
%			columns/abstand/.style={precision=2, zerofill, column name=\makecell{$Abstand$\\$(\pm 0.05)[mm]$} }, 
%			columns/n/.style={column name=\makecell{$n$\\$(Ordnung)$}, precision=0},
%			every head row/.style={before row=\hline,after row=\hline\hline},
%			every last row/.style={after row=\hline},
%			every first column/.style={column type/.add={|}{} },
%			every last column/.style={column type/.add={}{|} }
%			]{
%			abstand & n
%			12.9 & 1
%			24.45 & 2
%			37.40 & 3
%			49.35& 4
%			62.45 & 5
%			74.45 & 6
%			87.45 & 7
%			100.25 & 8
%			
%			}
%	\caption{Messwerte Einzelspalt}
%	\label{tab:werte_einzelspalt}
%\end{figure}
%


\section{Photovoltaische Solarzellen als Gleichstromquelle}
In diesem Experiment wird eine Solarzelle auf ihre Eigenschaften bei einer bestimmten Lichtintensität untersucht. Zu diesem Zweck wird eine Lampe in einem gewissen Abstand über einer Solarzelle befestigt. Bei konstantem Abstand und konstanter Lichtintensität kann mit einem regelbarem Widerstand (Abb. \ref{fig:schaltbild_solarzelle} $R_L$) eine Messreihe durchgeführt werden. Ziel ist es, die Werte für $I_{KS}$ den Kurzschlussstrom und $U_{LL}$ die Leerlaufspannung zu extrapolieren. \\
\begin{figure}[H]
	\centering
	\includegraphics[scale=0.6]{./figure/solarzelle_schaltplan.png}
	\caption{Schaltplan zur Bestimmung der Strom-Spannungskennlinie$^{[1]}$}
	\label{fig:schaltbild_solarzelle}
\end{figure}
\noindent
In Abbildung \ref{fig:schaltbild_solarzelle} ist der Aufbau ersichtlich. Durch Messung der Klemmenspannung ($I * R_I$) und der Gesamtspannung ($U$) lässt sich die Schaltung und somit die Solarzelle beschreiben.\\
Trägt man die Werte in einem U-I-Diagramm auf, erhält man die Strom-Spannungskennline der Solarzelle. Weiters kann über ein $\Omega-P$-Diagramm der Punkt der maximalen Leistung $P_{max}$ ermittelt werden.\\
Zuletzt wird der Kurvenfüllfaktor CFF über folgende Beziehung errechnet:
$$CFF = \frac{P_{max}}{I_{KS}*U_{LL}}$$
Der CFF-Wert bestimmt das Sprungverhalten von Spannung zu Leerlaufspannung. Bei einem Wert von CFF = 1 springt die Kurve ideal. Bei derzeitigen Optimierungen kann ein Wert von 0.8 - 0.9 erreicht werden.\\
Für die Berechnung des Stroms wird das Ohmsche Gesetzt verwendet:
$$I = \frac{U_{gemessen}}{R_I}$$
wobei $R_I$ in [1](1.3) gegeben ist.
\subsection{Messwerte und Ergebnisse}
\end{multicols}
\begin{figure}[H]
	\centering
	\includegraphics[scale=0.6]{./figure/solarzelle_pmax.png}
	\caption{Widerstand gegen Leistung; $P_{max}$}
	\label{fig:solarzelle_leistung}
\end{figure}
\begin{multicols}{2}
\noindent
Abstand zwischen Lichtquelle und Solarzelle:
$$l = 20cm$$
Werte mit SE-Mean:
$$I_{KS} = (61.58 \pm 1.99) mA$$
$$U_{LL} = (927.27 \pm 0.01) mV$$
$$P_{max} = (38.51 \pm 0.88) mW$$
$$CFF = 0.6744$$

\end{multicols}
\begin{figure}[H]
	\centering
	\includegraphics[scale=0.6]{./figure/solarzelle_strom_spannung.png}
	\caption{Spannung gegen Strom; Kennlinie}
	\label{fig:solarzelle_strom_spannung}
\end{figure}

\begin{multicols}{2}

\subsection{Diskussion}
Der Aufbau und  die Messung der Spannungen gestaltete sich dank zweier Fluke Messgeräte unproblematisch. Zu beachten war nur der korrekte Messbereich und das Umschalten auf Gleichstrom. Mit dem Schaltplan \ref{fig:schaltbild_solarzelle} ist der Aufbau bis auf den regelbaren (durch Drehung) Widerstand vollständig erklärt. \\
Beim regelbaren Widerstand musste vorsichtig gedreht werden und dabei auf den jeweiligen Anstieg bzw. Abfall wie in [1] (1.3) beschrieben zu achten. Bei der Auswertung wurde für die Berechnung der Werte $I_{KS}$ und $U_{LL}$ mit Origin 9 jeweils eine Interpolation in einem glatten Kurvenbereich durchgeführt. Für $U_{LL}$ ist der Fehler der interpolierten Werte mit 
$(0.0000774322) mV$ so gering das die Anzeigegenauigkeit der Messgeräte herangezogen wurde ($\pm 0.01mV$).\\
Bei der Interpolation für den Kurzschlussstrom $I_{KS}$ ergab sich durch die in \ref{fig:solarzelle_strom_spannung} ersichtlichen Knicke in der roten Kurve ein weit größerer Fehler. Die Ungenauigkeit könnte durch zu schnelle Messvorgänge hintereinander entstanden sein.\\
Der Kurvenfüllfaktor CFF liegt mit 0.6744 etwas unter den technisch machbaren Werten von 0.8-0.9. Dieser Wert kann aber als korrekt gewertet werden, da die blaue Kurve in \ref{fig:solarzelle_strom_spannung} einen runden Verlauf hat. Optimal wäre ein (fast) gerader Abfall.\\
Bei der Berechnung der Maximalen Leistung $P_{max}$ wurde ebenfalls eine Interpolation durchgeführt und beim Maximum eine punktuelle Koordinatenablesung durchgeführt.\\


%%%%%%%%%%%%%%%%%%%%%%%%%%%%%%%%%%%%%%%%%%%%%%%%
\section{Widerstandsbestimmung mittel Wheatstone-Brücke}

Eine Möglichkeit, einen Widerstand zu messen, ohne mit 2 Messgeräten (und 2 Messunsicherheiten) Spannung und Strom zu bestimmen ist eine Wheatstone-Brückenschaltung.
\\
Der zu messende Widerstand (hier $R_x$) und 3 andere werden, paarweise parallel geschaltet, und innerhalb beider Zweige in Serie. 

Die beiden Zweige der Parallelschaltung werden durch ein Voltmeter verbunden, wie in Abb. \ref{fig:wheatstone_schaltplan} skizziert.



\begin{figure}[H]
	\centering
	\includegraphics[scale=0.50]{./figure/wheatstone_schaltplan.png}
	\caption{Schaltskizze einer Wheatstonebrücke}
	\label{fig:wheatstone_schaltplan}
\end{figure}

An beiden Zweigen der Parallelschaltung liegt die gleiche Spannung an (Kirchhoff Schleifenregel). Diese teilt sich an den in Serie geschalteten Widerständen in derem Verhältnis auf.\\
Die Spannung im Messgerät ("Indikatorzweig") verschwindet genau dann, wenn die Verhältnisse der Widerstände an beiden Brückenzweigen gleich sind.

$$\frac{R_x}{R_0} = \frac{R_a}{R_b}$$
$$\Rightarrow R_x = R_0 \cdot \frac{R_a}{R_b}$$

Es genügt also, $R_0$ zu kennen, sowie das Verhältnis von $R_a$ zu $R_b$. Ihre Widerstandswerte sind nicht von Bedeutung.\\
\\
In diesem Versuch werden die WIderstände $R_a$,$R_b$ durch einen Draht realisiert, der an einer Längenskala befestigt ist, an der auch ein beweglicher Schleifer montiert ist. Durch diesen wird wird die Unterteilung in 2 Widerstände erzeugt (das Voltmeter ist am Schleifer angeschlossen).
\\
Da der Draht über die ganze Länge die gleiche Dicke hat, ist der Widerstand pro Längeneinheit überall gleich. Somit ist das Verhältnis der Drahtstücke, die durch den Schleifer getrennt werden gleichzeitig das Verhältnis der Widerstände.\\
Als $R_0$ wird eine Widerstandsdekade verwendet, auf der sich, in diskreten Schritten, Widerstände in von 1$ \Omega$ bis 10M$\Omega$ einstellen lassen
\\
Der Schleifer wird so lange verrückt, bis am Messgerät keine Spannung anliegt. Dann sind die Spannungsverhältnisse in beiden Zweigen gleich und der gesuchte Widerstand kann berechnet werden.\\
Es ist dabei eigentlich nicht notwendig, die Widerstandsdekade anders einzustellen. In der Diskussion wird jedoch näher erklärt, warum es sinnvoll ist, nicht das $R_a$-$R_b$-Verhältnis zu ändern, sondern $R_0$ anzupassen.

\subsection{Messwerte und Ergebnisse}

\subsection{Diskussion}
%%%%%%%%%%%%%%%%%%%%%%%%%%%%%%%%%%%%%%%%%%%%%%%%
\section{Reale Spannungsquelle}


\subsection{Messwerte und Ergebnisse}


\subsection{Diskussion}

%%%%%%%%%%%%%%%%%%%%%%%%%%%%%%%%%%%%%%%%%%

\section{Belasteter Spannungsteiler}
In diesem Experiment sollen 
\begin{figure}[H]
	\centering
	\includegraphics[scale=0.4]{./figure/spannungsteiler_schaltplan.png}
	\caption{Schaltplan Spannungsteiler a) unbelastet b) belastet$^{[1]}$}
	\label{fig:schaltbild_solarzelle}
\end{figure}
\noindent

\subsection{Messwerte und Ergebnisse}



\subsection{Diskussion}

%%%%%%%%%%%%%%%%%%%%%%%%%%%%%%%%%%%%%%%%%%


\section{Quellen}
$[1]$ Leitfaden, \url{http://www.univie.ac.at/anfpra/neu1/pw/pw9/PW9.pdf}\\
$[2]$ Quecksilberdampflampe, \url{http://de.wikipedia.org/wiki/Quecksilberdampflampe}\\
%$[2]$

\end{multicols}


\end{document}